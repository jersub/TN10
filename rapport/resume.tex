\section*{Résumé technique}

\paragraph{}
\asmile{} est une Société de Services en Ingénierie Informatique (SSII) française, dont l'activité principale repose sur le développement web et l'intégration de solutions open source.
\asmile{} est d'ailleurs aujourd'hui considéré comme le leader français dans le domaine.
Passionné par l'écosystème de l'open source, c'est naturellement que j'ai choisi d'intégrer cette entreprise pour effectuer mon projet de fin d'études.

\paragraph{}
Mon sujet de stage initial porte sur les problématiques d'intégration continue.
Comment tester en continu la qualité d'un développement logiciel ?
Quelles sont les outils permettant de construire une offre favoriserait l'émergence de cette qualité ?

En faisant intervenir mes connaissances concernant les bonnes pratiques du génie logiciel et en développant mes compétences système, j'ai pu tester ces outils comme Jenkins, Redmine ou Selenium et participer à leur mise en place chez des clients.

\paragraph{}
En parallèle, j'ai eu l'opportunité de travailler de bout en bout sur la mise en place d'une solution d'authentification open source particulière chez un client : \alinotp.
Elle fait intervenir les OTP, ou \etranger{One-Time Passwords}, qui sont en réalité des mots de passe éphémères générés par un matériel tiers.

\paragraph{}
En outre, j'ai pu améliorer d'autant plus mes compétences en travaillant sur de l'administration système interne ou en participant à d'autres projets clients, comme la mise en place d'une plateforme de supervision avec \acentreon{} ou en intervenant en tant que support sur un projet de développement d'application web.

\paragraph{Mots-clés}
\makeatletter
\@keywords
\makeatother

