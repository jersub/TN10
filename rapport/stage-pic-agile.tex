\section{Intérêt des méthodes agiles}

\paragraph{}
Les méthodes agiles sont très répandues dans les communautés du logiciel libre.
Les programmes y sont souvent développés en en suivant les principes.
Raison de plus pour que \asmile{}, leader sur le marché de l'open source, fasse la promotion de l'agilité après des entreprises.


\paragraph{}
Les méthodes agiles partent du constat que les spécifications initiales du client sont souvent \og volatiles \fg, dans le sens où elles évoluent relativement vite dans le temps et qu'elles souffrent régulièrement d'imprécision.
Les méthodes lourdes classique de développement impliquant spécifications, réalisation puis recette par le client se révèlent donc inadaptées.
Au contraire, il faut privilégier des cycles de développement courts qui apportent la flexibilité nécessaire pour pouvoir corriger rapidement à coût moindre les éventuelles erreurs de spécification ou de conception.

L'organisation du développement épouse alors deux aspects :

\begin{itemize}
	\item le côté itératif, car le logiciel est créé sur plusieurs cycles d'une durée relativement courte ;
	\item l'aspect incrémental, car chaque nouveau cycle améliore les fonctionnalités du précédent.
\end{itemize}


\paragraph{}
En outre, le Manifeste Agile~\cite{agile}, considéré comme l'acte généralisateur des méthodes agiles, invite à adopter un certain nombre de valeurs :

\begin{itemize}
	\item il faut privilégier les personnes et les interactions plutôt que de suivre aveuglément des procédures et d'être dirigé par ses outils ;
	\item la création régulière de jalons fonctionnels est une priorité pour garder cons\-tam\-ment à l'idée ce qu'est et ce que va être le produit final ;
	\item la clé est la collaboration étroite avec le client, alors que l'on a traditionnellement tendance à respecter un contrat et à suivre des spécifications initiales ;
	\item le client et le prestataire réalisateur doivent tous deux être flexible et accepter le changement.
\end{itemize}


\paragraph{}
Les méthodes agiles sont parfois difficiles à mettre en place car elles sont souvent en rupture avec les processus existants de l'entreprise.
De plus, la flexibilité requise peut être confondue avec un défaut d'organisation, ce qui amène inévitablement à des résultats catastrophiques.
Au contraire, de bons outils adaptés sont nécessaires pour effectuer un vrai suivi du projet.
Ils doivent permettre de tracer de façon transparente les changements et d'améliorer la communication entre les différents acteurs du projet.

