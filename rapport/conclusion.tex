\chapter{Bilan personnel}

\paragraph{}
J'ai beaucoup apprécié ce stage TN10 chez \asmile.
Le travail sur l'offre PIC avec mon maître de stage \agulet{} était très enrichissant, j'ai vraiment pu pousser les connaissances basiques que j'avais dans le domaine.
Notamment, j'ai pu acquérir une vision globale de l'offre d'intégration qui pouvait être proposée autour des concepts des méthodes agiles et de l'intégration continue.
Conseil, mise en place de processus de qualité et formation constituent la valeur ajoutée que peut apporter l'ingénieur.

Toutefois, je reste un peu déçu de ne pas avoir été confronté à un gros projet client de A à Z dans le domaine.

\paragraph{}
J'ai aussi beaucoup appris de ma mission d'intégration de solution OTP.
En contact direct avec le client et l'éditeur, j'ai à la fois pu mêler déploiement technique et gestion de projet.
À l'issue de ma formation à l'UTC et de ces six mois de stage de fin d'étude, c'est grâce à ces savoir-faire que je suis aujourd'hui devenu un ingénieur. 

\paragraph{}
Enfin, grâce aux plus petites missions système et au tâches de support projet, j'ai pu améliorer mes compétences techniques dans le domaine de l'administration système.

\paragraph{}
Malgré tous ces points positifs, j'ai décidé de ne pas travailler dans le domaine du système mais plutôt dans celui du développement.
Je pense y avoir un véritable potentiel que je prendrai plaisir à cultiver.
Le métier d'Ingénieur Études et Développement est donc un poste qui est plus en adéquation avec mes passions personnelles.

Mes compétences en ingénierie système restent néanmoins un véritable atout.
Cette compréhension globale des systèmes d'information, de la création logicielle à l'exploitation en passant par les mises en production, est une capacité que je pourrai mettre en avant dans ma carrière professionnelle.

