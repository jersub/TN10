\chapter{Déroulement du stage}

\section{Ma place au sein de la \abusys}

\paragraph{}
Passionné par l'écosystème de l'open source, c'est naturellement que j'ai choisi de postuler chez \asmile{} pour effectuer mon projet de fin d'études.
Sortant de la filière SRI de l'UTC, j'étais très intéressé de développer mes compétences système et c'est ainsi que j'ai pu intégrer la \abusys{}.

\paragraph{}
Dès mon premier entretien, j'ai défini avec \agulet{}, mon futur tuteur dans l'entreprise, le sujet principal de mon stage : nous avons choisi que je travaille sur les problématiques d'intégration continue.
En effet, à la \abusys{}, c'est une des spécialités de \agulet{}.
Il en est le référent quand il s'agit de fournir une prestation autour de ce domaine pour le compte de clients.
Pour ma part, c'est un sujet qui m'enthousiasme car il fait à la fois intervenir des connaissances en système et en qualité logicielle.
Ainsi, mon travail et ma réflexion sur l'intégration continue et ses outils sont développés dans le \refchap{pic}.

\paragraph{}
Par ailleurs, j'ai eu l'opportunité de travailler de bout en bout sur la mise en place d'une solution d'authentification open source particulière chez un client : \alinotp.
Elle fait intervenir les OTP, ou \etranger{One-Time Passwords}, qui sont en réalité des mots de passe éphémères générés par un matériel tiers.

Cette mission s'est déroulée sur la fin de mon stage, et j'ai alors eu l'opportunité de réaliser un véritable travail d'ingénieur système au même titre que mes collègues titulaires.
Elle est décrite en détail dans le  \refchap{linotp}.

\paragraph{}
En outre, j'ai eu l'occasion de travailler sur une multitude d'autres projets de taille plus ou moins importante :

\begin{itemize}
	\item mettre en place chez un client une solution de supervision \acentreon{} (\cfchap{centreon}) ;
	\item apporter un support d'administration système dit \emph{support projet} aux projets des différentes BU de développement de \asmile{} (\cfchap{support}) ;
	\item apporter un soutien au développement d'un projet client en retard pour \abt{} (développement avec le framework PHP Symfony et le framework JavaScript ExtJS) ;
	\item mettre en place une plateforme web de visioconférence pour \asmile{} avec la solution libre BigBlueButton ;
	\item auditer le code du logiciel libre Linbox-Converter pour le compte de Renault, qui permet de convertir des documents Microsoft Office vers d'autres formats ;
	\item chiffrer des projets système dans le cadre des processus d'avant-vente de \asmile.
\end{itemize}

\paragraph{}
Enfin, j'ai pu commencer certains travaux qui n'ont pas pu aboutir pour diverses raisons.
Ceux-ci sont décrits brièvement en \refsection{avortes}.



\section{Projets avortés}
\label{section:avortes}

TODO



\section{Planning effectif}

\paragraph{Février}
\begin{itemize}
	\item Support projet
	\item Mise en place d'un serveur de base de données Oracle
	\item Auto-formation sur la réplication de bases de données MySQL
	\item Auto-formation sur les outils Hudson/Jenkins et Selenium
	\item Travail sur l'offre \og Plateformes d'intégration continue \fg{} (PIC)
\end{itemize}

\paragraph{Mars}
\begin{itemize}
	\item Support projet
	\item Tentative de mise en place d'OpenMCU en interne
	\item Auto-formation sur l'ESB \apetals{}
	\item Développement sur le projet Abitbol
\end{itemize}

\paragraph{Avril}
\begin{itemize}
	\item Support projet
	\item Développement sur le projet Abitbol
	\item Développement Redmine
	\item Audit de code Linbox-Converter pour Renault
	\item Projet \acentreon{} pour \adacast{}
	\item Travail sur l'offre PIC
\end{itemize}

\paragraph{Mai}
\begin{itemize}
	\item Support projet
	\item Mise en place de BigBlueButton en interne
	\item Projet LinOTP pour la CNIL
	\item Travail sur l'offre PIC
\end{itemize}

\paragraph{Juin}
\begin{itemize}
	\item Support projet
	\item Fin du projet LinOTP pour la CNIL
	\item Rédaction d'une ébauche de livre blanc sur l'offre PIC
	\item Soutien au développement du projet de développement pour \abt{}
\end{itemize}

\paragraph{Juillet}
\begin{itemize}
	\item Soutien au développement du projet de développement pour \abt{}
\end{itemize}

