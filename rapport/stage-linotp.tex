\section{Mise en place d'une solution \alinotp{}}

\subsection{Principe des \etranger{One-Time Passwords} (OTP)}

TODO


\subsection{Contexte de la mission}

\paragraph{}
Au courant de l'année 2010, la Commission nationale de l'informatique et des libertés\footnote{La CNIL est une autorité administrative indépendante française.Elle est chargée de veiller à ce que l'informatique soit au service du citoyen et qu'elle ne porte atteinte ni à l'identité humaine, ni aux droits de l'homme, ni à la vie privée, ni aux libertés individuelles ou publiques.\cite{cnil}} (CNIL) décide d'ouvrir son portail \aintranet{} pour certains de ses utilisateurs : les commissaires.
Au nombre d'une vingtaine, ceux-ci ont besoin d'accéder au contenu du site web à l'extérieur de leur lieu de travail, à la maison par exemple.

\paragraph{}
La CNIL avait déjà fait appel aux prestations de \asmile{} pour bénéficier d'un support ponctuel sur \atypo{}, le système de gestion de contenu utilisé pour le portail.
C'est pour cela qu'elle a également choisi \asmile{} pour mettre en place l'architecture permettant un tel accès depuis l'extérieur.

C'est la CNIL elle-même qui a choisi le type d'autentification à utiliser : les OTP. \asmile{} a alors proposé fin 2010 l'utilisation d'une solution de la marque RSA, leader sur ce marché.
Pour des raisons qui sont restées inconnues -- le nombre de licences commandées était trop faible ? -- RSA n'a pas daigné répondre aux commandes de licences et de matériel.

Le retard sur la mise en place de l'architecture a alors poussé \asmile{} à changer de solution pour se tourner vers une alternative open source : \alinotp{}.


\subsection{La solution \alinotp{}}

TODO


\subsection{Ma démarche}

TODO


\subsection{Architecture mise en place}

TODO

